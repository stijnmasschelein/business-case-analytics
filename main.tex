\documentclass{article}
\usepackage[hybrid]{markdown}
\usepackage[utf8]{inputenc}
\usepackage{dirtytalk}
\usepackage{fancyhdr}
\usepackage{datetime}
\usepackage{titlesec}
% Fonts
\usepackage{ebgaramond}
\usepackage[varqu,varl,var0,scaled=0.97]{inconsolata} 
\usepackage{FiraSans}

\title{Business Case Data Analytics}
\author{stijnmasschelein }
\date{December 2019}

\pagestyle{fancy}
\fancyhead[l]{\inserttitle}
\fancyhead[R]{Stijn Masschelein}
\fancyfoot[c]{\thepage}

\makeatletter
\let\inserttitle\@title
\makeatother

\begin{document}
\begin{markdown}
# Synopsis

The goal of this iniative is to create an informal incubator for data analytics projects. undergraduate students can work on projects to develop data analysis, programming, and project managment skills. initially students can work on small research projects or with university data. These projects should prepare them to work as interns with industry partners. The long term reward for us would be that our students can take on programming tasks that we do not have to do and they are involved with the latest data analytics projects in industry. 


# Philosophy

I want to start this initiative from a couple of principles. The main one is that 

Focused on open source languages like `R` and `Python` and data management and data visualisation + Building tools to manage . Writing shall be done in `overleaf` or `markdown` documents. Maybe contact former student about open data and legal issues. Document everything.

# Type of Projects
- See notes on booknote
- Ray suggest to use internal budgeting data and maybe student numbers.     There are going to be issues with this. 
- Help with funding
- Merging data on publications 

# Example

Melinda Hodkiewtiz and Data Safety Lab

# Outcomes for students

## Recognition
- Working on proto types, they keep the intellectual property, opportunity to have a portfolio.

## Outside opportunities at the end 
- Landgate (for PhD students)
- Western Power
- Placements 
- Ceed funding engineering

# Timeline
## First semester 2020
- Attracting students for projects. First attempts for some projects.
- Attracting projects from within the university. Not RA jobs.
- Sorting out legal IP issues.
- Thinking of funding opportunities
- Two weekly Friday (?) progress meetings

## Second semester 2020
- Evaluating the first semester and looking for funding
- Adapt governance structure

## First semester 2021 
- Put structure in place with student leaders. PhD students are more        experienced. We should help with developing some team leader skills but it can also be part of the documentation of the process. 

## Second semester 2021 

\end{markdown}
\end{document}
